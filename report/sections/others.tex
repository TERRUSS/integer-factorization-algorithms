\section{Autres algorithmes}
    \subsection{Algorithme du Crible Algébrique}
    
        C'est aujourd'hui l'algorithme le plus efficace pour la factorisation en facteurs premier pour un grand nombre.
        Les derniers "records" de factorisation sonts tous faits à partir de cet algorithmes, et recemment il a été capable de factoriser un nombre RSA-240 (ie. un nombre de 240 bits).
        
        Le Crible Algébrique étant la méthode de référence, c'est sur sa complexité que l'on fixe la longueur des clefs RSA pour garder un niveau de securité.
        
        L'algorithme étant trop complexe et mathématiquement poussée, il ne m'était pas possible de l'impémenter pour ce projet. Voici cependant les étapes de la méthode :
        
        \begin{enumerate}
            \item Création de polynomes en rapport avec les entrées
            \item Test et selection d'un des polynomes
            \item Trouver des "relations" algebriques et relationelle avec des nombres friables en rapport avec la base factorielle du polynome.
            \item Création et résolution d'une large matrice pour trouver les les sets de relation dont le produit des normes de ce set sont un carré parfait.
            \item Trouver la racine carrée des polynomes sur un corps fini
            \item Executer le Théorème des Restes Chinois
            \item et enfin tester les resultats produits.
        \end{enumerate}