
%-------------------------------LATEX SYNTAX----------------------------------
%
%	enter:				\\, \linebreak, \newline
%	new page: 			\newpage
%	tab:				\tab
%	new section:		\section{name} text
%	new paragraph		\subsection{name} text	(subsub...section{name} for more layers)
%	refer:				\nameref{label name}	(place "\label{name}" at the position you want to refer to)
%	%, &, $, etc:		\%, \&, \$, etc		(these are latex operators, add a "\" to type it as text)
%	add comment:		\commred{text}, \commblue{text}, \commpurp{text}, \commgreen{text}
%	bullet points:		\begin{itemize} \item{text} ... \item{text} \end{itemize}
%	clean code:			\cleancode{text}
%	idem without indent:\cleanstyle{text}
%	bold, italic, under:\textbf{text}, textit{text}, \underline{text}
%	table:				\begin{tabular}{c c c} text \end{tabular}	('&' for tab, '\\' for new line)
%	
%	use Google for the rest
%	
%------------------------------------------------------------------------------

\documentclass[a4paper]{article}

\usepackage[utf8x]{inputenc}
\usepackage[french]{babel}
\usepackage{amsmath}
%\usepackage{titlesec}
\usepackage{color}
\usepackage{graphicx}
\usepackage{fancyref}
\usepackage{hyperref}
\usepackage{float}
\usepackage{scrextend}
\usepackage{setspace}
\usepackage{xargs}
\usepackage{multicol}
\usepackage{nameref}
\usepackage[pdftex,dvipsnames]{xcolor}
\usepackage{sectsty}


\usepackage{listings}
\lstset{
basicstyle=\ttfamily,
frame=single
}

\definecolor{nordBg}{HTML}{dddddd}
\definecolor{nordFg}{HTML}{2e3440}
\definecolor{nordBlack}{HTML}{3b4252}
\definecolor{nordRed}{HTML}{bf616a}
\definecolor{nordGreen}{HTML}{a3be8c}
\definecolor{nordYellow}{HTML}{ebcb8b}
\definecolor{nordBlue}{HTML}{81a1c1}
\definecolor{nordMagenta}{HTML}{b48ead}
\definecolor{nordCyan}{HTML}{88c0d0}
\definecolor{nordWhite}{HTML}{e5e9f0}


\graphicspath{ {images/} }
\newcommand\tab[1][1cm]{\hspace*{#1}}
\interfootnotelinepenalty=10000
%\titleformat*{\subsubsection}{\large\bfseries}
\subsubsectionfont{\large}
\subsectionfont{\Large}
\sectionfont{\LARGE}
\definecolor{cleanOrange}{HTML}{D14D00}
\definecolor{cleanDark}{HTML}{2E3440}
\definecolor{cleanYellow}{HTML}{FFFF99}
\definecolor{cleanBlue}{HTML}{3d0099}
%\newcommand{\cleancode}[1]{\begin{addmargin}[3em]{3em}\fcolorbox{cleanOrange}{cleanYellow}{\texttt{\textcolor{cleanOrange}{#1}}}\end{addmargin}}
\newcommand{\cleancode}[1]{\begin{addmargin}[3em]{3em}\texttt{\textcolor{nordBlack}{#1}}\end{addmargin}}
\newcommand{\cleanstyle}[1]{\text{\textcolor{nordBlack}{\texttt{#1}}}}

\definecolor{red}{HTML}{bf616a}
\definecolor{blue}{HTML}{d08770}
\definecolor{OliveGreen}{HTML}{a3be8c}
\definecolor{Plum}{HTML}{b48ead}

\usepackage[colorinlistoftodos,prependcaption,textsize=footnotesize]{todonotes}
\newcommandx{\commred}[2][1=]{\textcolor{Red}
{\todo[textcolor=nordFg,linecolor=red,backgroundcolor=nordBg,bordercolor=red,#1]{\texttt{#2}}}}
\newcommandx{\commblue}[2][1=]{\textcolor{Blue}
{\todo[textcolor=nordFg,linecolor=blue,backgroundcolor=nordBg,bordercolor=blue,#1]{\texttt{#2}}}}
\newcommandx{\commgreen}[2][1=]{\textcolor{OliveGreen}{\todo[textcolor=nordFg,linecolor=OliveGreen,backgroundcolor=nordBg,bordercolor=OliveGreen,#1]{\texttt{#2}}}}
\newcommandx{\commpurp}[2][1=]{\textcolor{Plum}{\todo[textcolor=nordFg,linecolor=Plum,backgroundcolor=nordBg,bordercolor=Plum,#1]{\texttt{#2}}}}


\lstset{frame=Trbl,numbers=left}

%-----------------------------------------BEGIN DOC----------------------------------------

\begin{document}

\pagecolor{nordBg}
\color{nordFg}

\title{
\hline \vspace{.3cm}
{\Huge Analyse d’algorithmes \\et validation de programmes
\vspace{.3cm} \hline \vspace{1cm}
{\large\linebreak\\}}{\Large Projet
\\\linebreak\linebreak}
\linebreak{\Huge[ISTY]}
\linebreak{\small - UVSQ, Paris Saclay -\linebreak}}
\author{\\Olivier Benaben\\
\\\\
\\
\\\\
\\
\\
}
\date{15 Février 2021}
\maketitle
\newpage


%----------------------------------------ABSTRACT-----------------------------------------

\section{Introduction}\label{Introduction}%------------------------------
    \subsection{Choix du problème}
    
        Pour ce projet j'ai choisis de m'intéresser aux \cleanstyle{algorithmes de décomposition}
        \\\cleanstyle{ de nombres en produit de facteurs premiers}.\\
        
        Ce choix vient du fait que ces algorithmes sont très importants ajourd'hui par la place stratégique qu'ils occupent dans le domaine de la cryptographie.\\
        En effet le chiffrements asymétriques le plus connu et le plus utilisé - le \textit{RSA} - consiste à générer ses clefs à partir de deux très grands nombres premiers, qui sont multipliés ensuite (on appelle cette partie de la clef $N$).\\
        Une attaque du RSA consisterait donc à retrouver les deux premiers $p$ et $q$ tels que $N = p \times q$, et c'est là qu'interviennent les algorithmes de décomposition en produit  de facteurs premiers.\\
        
        Les meilleurs algorithmes actuels résolvant ce problème sur des machines "normales" (non-quantiques) sont \textit{sous-exponentiels} mais  aussi \textit{super-polynomiaux}. C'est clairement un problème appartenant à la classe $NP$ : il est suffisement compliqué, mais en vérifier une solution est très simple (il suffit en effet de multiplier les solutions et d'en vérifier l'égalité avec le nombre en entrée.\\
        En revanche on peut trouver l'algorithme de Shor qui, executé sur un ordinateur quantique, arrive à le résoudre en temps polynomial !\\
        Le plus grand nombre factorisé par ce type de marchines s'élève cependant à ce jour à $35 = 7 \times 5$.
        
        Dans ce devoir nous comparerons trois algorithmes de décompositions en facteurs premiers : une première approche par Division Successive, 


    \vspace{1cm}
    \tableofcontents
    
    %------------------------------
    
\section{Méthode par Divisions Successives}

    C'est l'algorithme le plus simple et le plus evident auquel on peut penser pour factoriser un nombre en facteurs premiers. C'est aussi un algorithme glouton, comme on peut voir a la lecture du code :
    
    \begin{lstlisting}
func trial_division(n int) []int {
  var a []int
  var f = 2

  for n > 1 {
    if n % f == 0 {
      a = append(a, f)
      n /= f
    } else {
      f += 1
    }
  }

  return a
}
    \end{lstlisting}
    
    \subsection{Principe}
    Cet algorithme a ete ennonce par Fibonacci de la maniere suivante :
    
    On test pour tous les entiers premiers $p_i$ entre 2 $n$ si ce $n$ peut etre divise par le $p_i$. En trouvant un premier $p_p$ qui divise $n$, on trouve automatiquement le $p_q$ associe tel que : $p_p \cdot p_q = n$.
    On se retrouve avec une liste de nombres qui divisent $n$. On selectionne alors les deux plus grands 
    
    \subsection{Complexité}
    
    L'algorithme n'utilise que les variables $A$ et $b$ (on considerera les variables $i$, $j$ et $k$ commes negligeables).\\
    La complexite en espace de l'algorithme est donc : $$\lvert A \rvert + \lvert b \rvert = n \cdot n + n = n ^2 +n$$
    
    \subsection{Améliorations}
    
    Wheel Factorisation
    
    %------------------------------

\section{Méthode du Crible Rationnel}
    \begin{lstlisting}
#include <stdio.h>
#include <stdlib.h>
#include <omp.h>
#include <time.h>

double f (double x) {
  return 4/(1 + x*x);
}
    \end{lstlisting}
    

    \subsection{Complexité}
        
        L'algorithme n'utilise que les variables $A$ et $b$ (on considerera les variables $i$, $j$ et $k$ commes negligeables).\\
        La complexite en espace de l'algorithme est donc : $$\lvert A \rvert + \lvert b \rvert = n \cdot n + n = n ^2 +n$$
    
    \subsection{Améliorations}
    
    Special number field sieve, General number field sieve

    %------------------------------
    
\section{Méthode de Fermat}
    \begin{lstlisting}
#include <stdio.h>
#include <stdlib.h>
#include <omp.h>
#include <time.h>

double f (double x) {
  return 4/(1 + x*x);
}
    \end{lstlisting}
    

    \subsection{Complexité}
        
        L'algorithme n'utilise que les variables $A$ et $b$ (on considerera les variables $i$, $j$ et $k$ commes negligeables).\\
        La complexite en espace de l'algorithme est donc : $$\lvert A \rvert + \lvert b \rvert = n \cdot n + n = n ^2 +n$$
    
    \subsection{Améliorations}
    
    Special number field sieve, General number field sieve


     %------------------------------
\section{Autres algorithmes}
    \subsection{Algorithme du Crible Algébrique}
    
        C'est aujourd'hui l'algorithme le plus efficace pour la factorisation en facteurs premier pour un grand nombre.
        Les derniers "records" de factorisation sonts tous faits à partir de cet algorithmes, et recemment il a été capable de factoriser un nombre RSA-240 (ie. un nombre de 240 bits).
        
        Le Crible Algébrique étant la méthode de référence, c'est sur sa complexité que l'on fixe la longueur des clefs RSA pour garder un niveau de securité.
        
        L'algorithme étant trop complexe et mathématiquement poussée, il ne m'était pas possible de l'impémenter pour ce projet. Voici cependant les étapes de la méthode :
        
        \begin{enumerate}
            \item Création de polynomes en rapport avec les entrées
            \item Test et selection d'un des polynomes
            \item Trouver des "relations" algebriques et relationelle avec des nombres friables en rapport avec la base factorielle du polynome.
            \item Création et résolution d'une large matrice pour trouver les les sets de relation dont le produit des normes de ce set sont un carré parfait.
            \item Trouver la racine carrée des polynomes sur un corps fini
            \item Executer le Théorème des Restes Chinois
            \item et enfin tester les resultats produits.
        \end{enumerate}
    
    
\section{Ressources}%------------------------------

Wikipedia\\
\url{https://www.wikiwand.com/en/Integer_factorization}\\
\url{https://www.wikiwand.com/en/Dixon%27s_factorization_method}\\
\url{https://www.wikiwand.com/en/Quadratic_sieve}\\
\url{https://www.wikiwand.com/en/General_number_field_sieve}\\
\url{https://www.wikiwand.com/en/Fermat%27s_factorization_method}\\
\url{https://www.wikiwand.com/en/RSA_(cryptosystem)}\\
\url{https://www.wikiwand.com/en/Shor%27s_algorithm}\\

RSA Cracking Challenge\\
\url{https://medium.com/asecuritysite-when-bob-met-alice/cracking-rsa-a-challenge-generator-2b64c4edb3e7}\\

Crible Algébrique\\
\url{https://github.com/AdamWhiteHat/GNFS/blob/master/Integer%20Factorization%20-%20Master%20Thesis%20-%20Per%20Leslie%20Jensen.pdf}\\
\url{https://github.com/AdamWhiteHat/GNFS/blob/master/The%20Multiple%20Polynomial%20Quadradic%20Sieve%20-%20R.D.%20Silverman.pdf}

\end{document}
